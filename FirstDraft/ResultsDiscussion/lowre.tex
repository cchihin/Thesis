\chapter{Impact of shear on spiral defect chaos and ISRs (12 Aug - 31 Oct)}\label{chap:lowre}

\section{The impact of shear on spiral defect chaos (1 month)}
HYPOTHESIS: SDC breakdown at $Re \sim 1 - 10$, which increases as $Ra$ increases.
\begin{enumerate}
    \item Present 3 (or more?) elementary states for $Ra = 3000$.
    \item Show that they form the state-space backbone of SDC.
    \item Performing $Re-$branch continuation studies for these elementary states.
    \item Branches may disappear/distabilise at $Re = 1$ for $Ra = 3000$.
    \item Repeat for higher $Ra = 5000, 8000$. Should we consider $Ra = 8000$? Note that oscillatory instability kicks in at $Ra \sim 5200 - 8540$ (Bodenchatz 2000 fig.6)
\end{enumerate}

\section{Influence of $Re$ on the orientation and wavelengths of rolls (1 month)}
HYPOTHESIS: Longitudinal rolls are the preferred orientation. Stability boundaries of the Busse balloon expands with increasing $Re$.
\begin{enumerate}
    \item Perform $Re-$branch continuation studies for transverse, longitudinal and oblique rolls.
    \item Branch will disappear/destabilise for transverse/oblique rolls at $Re \sim 10-100$.
    \item Perform linear stability analysis of longitudinal rolls at $Re = 0 - 100$.
\end{enumerate}

