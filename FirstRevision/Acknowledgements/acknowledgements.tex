\chapter*{Acknowledgements}
There are many people who have shaped my experience as a PhD student whom I would like to express my gratitude for.
First and foremost, I want to express my deepest gratitude to my supervisors, Spencer and Yongyun.
I have learned and grown a lot under their tremedously kind and patient supervision, for allowing me to explore research ideas in shark skins and hydrodynamic instabilities, to contributing to the large codebase of Nektar++, and at times canning ideas that did not work! (Hint: the topic that did not make it to this thesis).
I will miss the our constant weekly catch-ups and engaging deep-dives.
Apart from the PhD, they also openly supported personal development opportunities, allowing me to pursue interests such as the Nektar++ redesign project, engaging in graduate teaching assistant roles, pursuing a summer research exchange at LadHyX and an internship at IBM Research, significantly enhancing the PhD experience which I am deeply grateful for.

I would like to thank Yongyun, who also happens to be my MSc thesis supervisor, for introducing the topic of reduced-order modelling of turbulent Couette flows as part of my MSc thesis. 
Perhaps more importantly, for giving me the confidence I needed to pursue a PhD program at Imperial.

I would also like to thank Lutz, for his kind hospitality, ensuring that I was comfortable at LadHyX in the summer of 2024. Despite being a short three months stint, I learned about advanced modal decomposition techniques and was exposed to the broad fluid mechanics research activities at LadHyX.

I would also like to thank Basman, my Bachelor's thesis supervisor at Nanyang Technological University (NTU), during which we worked on humpback whale leading-edge turbercles. 
I believe it is fair to say that he was partly responsible for planting the "seed" for my fascination for computational fluid dynamics.

I would like to also thank Eloisa and Geeth, my project supervisor and manager at IBM Research, for giving the opportunity to experience, learn and contribute to AI research activities within a corporate company.

I would also like to thank the Nektar++ community, particularly Chris, Jacques and Dave for their patient guidance on how to do contribute to Nektar++ properly.

I would also like to thank my collaborators, Mohammad and Raj, for providing guidance and advice during the PhD and also providing the opportunity to work on physical informed neural networks. 

The PhD experience isn't complete without the camaraderie between friends which include, unwinding over a cheeky pint on Friday evenings, Mario karting in the common room, going to Sagar and exploiting its 50\% discount, honest burgers student discounts, going to Wasabi at South Kensington at closing to get half-priced bentos, countless muffin and coffee breaks, peaceful walks around campus, Hyde park runs, booking the swimming pool sessions just to use the sauna, many gym/swim/badminton/climbing sessions, appearing decently surprise to bump into each other in the office on the weekends, and bantering over the weather, reviewers and PhD life. 
I will forever cherish these unique experiences together with Henrik, Parv, Alex, Kaloyan, Ganlin, Steffi, Lidia, Christian, Kazuki, Priyam, Sid, Elise, Yu Hang, Cheng Wei, Zhao, Zilin, Victor, Mohsen, Guglielmo, Jo\~{a}o, Yacine and many others.

The person who probably deserves the most credit would be my fianc\'{e}e, Angelica, who have been there atthe beginning, ready to lend a listening ear to my monologue on how cool CFD is.
The distance between Singapore and London over the years have been tough, and I am happy that we will be closing the gap and I am looking forward to our shared future.
Last but not least, I am grateful to my loving parents and family, for supporting my academic interests aboard despite the distance. 

This thesis is dedicated to everyone who have impacted my life.
