\chapter{Conclusions}\label{chap:5}

In our concluding chapter, we emphasise the main contributions of this thesis and provide recommendations for possible avenues of future work.
We explored the transitional regimes of Rayleigh-B\'{e}nard Poiseuille (RBP) flows arising from the combination of shear and buoyancy forces in \S \ref{chap:3}.
The main findings can be summarised in the following,
\begin{enumerate}
    \item The existence of the \textit{thermally-assisted sustaining process}, providing an alternative pathway towards turbulence via linearly unstable longitudinal rolls in confined domains.
    \item At moderate Reynolds number, a new spatio-temporal regime referred to as intermittent rolls was identified. It is characterised by the quasi-cyclic breakdown and regeneration of longitudinal rolls for moderate Reynolds numbers in large spatial domains.
    \item At low Reynolds number, a buoyancy dominant regime is primarily influenced by the magnitude of the Rayleigh number and is independent of Reynolds number. In this regime, the convection structures are consistent regimes in Rayleigh-B\'{e}nard convection, such as the bistable system between spiral-defect chaos (SDC) and ideal-straight rolls (ISRs), and the presence of oscillatory instabilities. These structures appear to be simply convected along the mean flow.
\end{enumerate}

Next, we provide possible avenues of future work for transitional RBP, which may include
\begin{enumerate}
    \item Performing a detailed bifurcation analysis of invariant states arising around the thermally-assisted sustaining process (see figure \ref{fig:compiled-full}) in confined domains and minimal band units \citep{tuckerman_patterns_2020}, possibly identifying complex global bifurcations.
    \item Conducting bifurcation analysis necessitates the implementation of Jacobian-free Newton-Krylov (JFNK) methods in Nektar++ \citep{knoll_jacobian-free_2004}. Variants of the JFNK method includes a hook-step modification \citep{viswanath_recurrent_2007} or the utilisation of preconditioners \citep{yan_nektar_2021} may accelerate convergence. Apart from the JFNK method and its variants, a recent adjoint-based variational method may offer better convergence properties for computing periodic orbits \citep{parker_variational_2022}.
    \item The extension of the \textit{thermally-assisted sustaining process} to other buoyancy and shear driven systems such as the Rayleigh-B\'{e}nard Couette (RBC) flow, which may be easier for experimental verification in the absence of a mean flow.
    \item It may be worth exploring the influence of Reynolds number on bistability between spiral-defect chaos (SDC) and ideal straight rolls (ISRs) in detail, and the secondary linear stability of longitudinal rolls. However, we caution that there may be numerous invariant states in the buoyancy-driven regime as witnessed in \S \ref{chap:4}.
\end{enumerate}

In \S \ref{chap:4}, we examined the state space structure of the bistability between SDC and ISRs in Rayleigh-B\'{e}nard convection in various domain sizes. 
The main contributions of the study are as follows,
\begin{enumerate}
    \item By systematically decreasing the spatial domain to a minimal domain, we have identified stable invariant solutions referred to as \textit{elementary} states. These elementary states are visibly (see figure \ref{fig:fig1}) and statistically (see figure \ref{fig:averagedProfile}) similar to the spatially localised patterns of SDC, underpinning the pattern formation of SDC.
    \item The identification of multiple edge states (and likely many more), forming local attractors along the edge between the basin boundary of attraction of ISRs and elementary states.
    \item Multiple networks of heteroclinic orbits (and likely many more) between stable and unstable ISRs.
    \item The identification of unstable ISRs far from the boundaries of the Busse balloon, which sit on the boundary between stable ISRs and SDC.
\end{enumerate}
While the newly identified elementary states resemble the localised features of SDC, they remain stable and time independent and therefore do not capture the temporal dynamics of SDC. 
Possible avenues of future research may be dedicated to the following,
\begin{enumerate}
    \item Computing the unstable invariant solutions of Rayleigh-B\'{e}nard convection, such as their equilibria, periodic orbits and heteroclinic connections between them, potentially offering descriptions of the temporal dynamics of SDC (necessitating the implementation of bifurcation tools in Nektar++ mentioned above).
    \item The onset of sustained spiral defect chaos appears to be linked to a critical domain size, below which SDC remains transient. Therefore, a bifurcation analysis of elementary states with respect to the spatial domain size could be performed to determine the critical domain size for the onset of sustained chaos.
    \item By performing bifurcation analysis in the regime of transient SDC, we may draw similar ideas from transient turbulence (see \ref{sec:bkgrd_nondysys}), perhaps revealing global bifurcations such as crisis bifurcations where the chaotic attractor becomes leaky or homoclinic tangles in Rayleigh-B\'{e}nard convection.
\end{enumerate}