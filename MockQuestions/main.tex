\documentclass{article}
\usepackage[2cm]{geometry}
\begin{document}
\section*{Question 1 - Big picture}
\subsection*{In one or two minutes, can you explain the \textit{core contribution} of your PhD?}
\begin{enumerate}
    \item What problem did you tackle?
    \item Why is it important scientifically or practically?
    \item What did you do that others had not done before?
\end{enumerate}

My PhD thesis investigates the transitional regime of fluid motions driven by both shear and buoyancy forces, a regime that is unexplored compared to purely-shear or buoyancy driven flows.
These flows are ubiquitous in applications such as electronic chip cooling, chemical vapour deposition, and geophysical and atmospheric systems, where shear and buoyancy forces interact.

There are two key contributions to this thesis. Firstly, the interaction between shear and buoyancy forces give rise to five distinct fluid motions across a range of Reynolds and Rayleight numbers considered. The intermittent regime represents a newly identified regime characterised by longitudinal rolls which spontaneously laminarise before being regenerated again. In a small domain, this give rise to interesting dynamics such as the thermally-sustainied turbulent process.

The second contribution is dynamical systems interpretation of the bistable attractor between the stable convection rolls and chaotic rolls known as spiral defect chaos. As we observed that patterns are repeated in spiral defect chaos, we minimise the domain to isolate such patterns. To our surprise, these patterns remain stable and underpinn the pattern formation of SDC. We then systematically investigated the state space structure of the system, uncovering edge states, heteroclininc orbits and pathways towards SDC via a secondary instability.

\section*{Question 2 - Physical understanding}
\subsection*{Your work focuses on transition and hydrodynamic instability}
Can you explain:
\begin{enumerate}
    \item The physical mechanism behind the instability or transition you study,
    \item Why \textit{linear stability theory} alone is insufficient to explain the observed transition?
\end{enumerate}
\begin{enumerate}
    \item a) Shear, give rise to a mean velocity gradient which allows perturbations to extract energy from. The balance between inertial and viscosity known as the Reynolds number then determines if the perturbations may grow or decay.
    \item b) Buoyancy instability due to the temperature gradient which allows pertubrations to extract energy from. The balance between temperature gradient and the kinematic viscosity and thermal diffusivity then determines if fluid motion occurs, leading to the onset of buoyancy-driven convection.
\end{enumerate}
Linear stability theory predicts the asymptotic fate of infinitesimal perturbations.
However, in many shear-driven, transition occurs through finite amplitude disturbances or transient growth mechanisms, meaning turbulence can be observed even when the laminar base flow is linearly stable.
Such mechanism are non-modal growth, lift-up effect and Orr-mechanism.

\section*{Question 3}
\end{document}
